
        \documentclass{article}
        \usepackage[utf8]{inputenc}
        \usepackage[T2A]{fontenc}
        \begin{document}
Найдем производную функции:
Возьмем простенькую производную
Легко заметить, что

        \begin{equation}
\frac{\partial}{\partial x}(x^{2}\cdot x^{2})
        \end{equation}
После сокращения единиц, получаем

        \begin{equation}
\frac{\partial}{\partial x}(x^{2})\cdot x^{2}+x^{2}\cdot \frac{\partial}{\partial x}(x^{2})
        \end{equation}
Вспомним числа Каталана

        \begin{equation}
\frac{\partial}{\partial x}(x^{2})=x^{2}\cdot (\frac{\partial}{\partial x}(2)\cdot \ln{x}+\frac{2\cdot \frac{\partial}{\partial x}(x)}{x})
        \end{equation}
Легко увидеть принцип Дирихле

        \begin{equation}
\frac{\partial}{\partial x}(2)=0
        \end{equation}
Из элементарных свойств конических сечений получим

        \begin{equation}
\frac{\partial}{\partial x}(x)=1
        \end{equation}
Используя метод электростатических изображений

        \begin{equation}
\frac{\partial}{\partial x}(x^{2})=x^{2}\cdot (\frac{\partial}{\partial x}(2)\cdot \ln{x}+\frac{2\cdot \frac{\partial}{\partial x}(x)}{x})
        \end{equation}
Из диаграммы Эйлера

        \begin{equation}
\frac{\partial}{\partial x}(2)=0
        \end{equation}
Из геометрических соображений получим

        \begin{equation}
\frac{\partial}{\partial x}(x)=1
        \end{equation}
Вспомним числа Каталана

        \begin{equation}
x^{2}\cdot (0\cdot \ln{x}+\frac{2\cdot 1}{x})\cdot x^{2}+x^{2}\cdot x^{2}\cdot (0\cdot \ln{x}+\frac{2\cdot 1}{x})
        \end{equation}
После элементарных преобразований получаем:
Когда-то существовал анекдот, напоминающий следующую формулу

        \begin{equation}
x^{2}\cdot \frac{2}{x}\cdot x^{2}+x^{2}\cdot x^{2}\cdot \frac{2}{x}
        \end{equation}
После :
Из диаграммы Эйлера

        \begin{equation}
x^{2}\cdot \frac{2}{x}\cdot x^{2}+x^{2}\cdot x^{2}\cdot \frac{2}{x}
        \end{equation}
Где

        \end{document}
